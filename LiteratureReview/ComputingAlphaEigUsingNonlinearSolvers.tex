\documentclass{article}
\usepackage[utf8]{inputenc}
\usepackage{notes}

% In your .tex file
% !TEX program = pdflatex

\newcommand{\thiscoursecode}{[XXXX] (\#\#\#)}
\newcommand{\thiscoursename}{Alpha-Eigenvalue Research at UC Berkeley}
\newcommand{\thisprof}{Dr. Great Professor}
\newcommand{\me}{Liam Horne}
\newcommand{\thisterm}{Paper Summary}
\newcommand{\website}{MYWEBSITE.COM}

% Headers
\chead{\thiscoursename}
\lhead{\thisterm}


%%%%%% TITLE %%%%%%
\newcommand{\notefront} {
\pagenumbering{roman}
\begin{center}

{\ttfamily \url{\website}} {\small}

\textbf{\Huge{\noun{\thiscoursecode}}}{\Huge \par}

{\large{\noun{\thiscoursename}}}\\ \vspace{0.1in}

  {\noun \thisprof} \ $\bullet$ \ {\noun \thisterm} \ $\bullet$ \ {\noun {University of Waterloo}} \\

  \end{center}
  }
  
  \title{Computing the Alpha-Eigenvalue Using Nonlinear Solvers}
  
\date{\vspace{-5ex}}
%\date{}

\begin{document}

\maketitle

\section*{Information}

Authors: Erin D. Fichtl and James S. Warsa \\

Organization: Los Alamos National Laboratory, Computational Physics Group

\section*{Introduction}

$\alpha$-eigenvalue formulation of the multigroup (MG) neutron transport equation (NTE) formed by assuming exponential time-behavior of flux

\begin{equation*}
\psi_{g,n}(\mathbf{r},t) = e^{\alpha t} \psi^{\alpha}_{g,n}(\mathbf{r})
\end{equation*}

Plugging in the flux to the MG NTE:

\begin{equation*}  
\bigg ( \frac{\alpha}{v_g} + \hat{\Omega}_n \cdot \nabla + \sigma_{t,g}(\mathbf{r}) \bigg ) \psi^{\alpha}_{g,n} = \sum_{g'} \sigma_{s,g'\rightarrow g}(\mathbf{r})\phi^{\alpha}_{g'} + \sum_{g'} \bar{\nu} \sigma_{f,g' \rightarrow g}(\mathbf{r})\phi^{\alpha}_{g'}
\end{equation*}

Dominant mode corresponds to algebraically largest eigenvalue.

\section*{Methods}

\subsection*{k-eigenvalue method}

Traditionally computed using multiple k-eigenvalue solutions.

Limitation: If $\alpha$ is negative enough, effective total cross section becomes negative in some groups: 

\begin{equation*}
\tilde{\sigma}_{t,g} = \bigg ( \frac{\alpha}{v_g} + \sigma_{t,g} \bigg ) < 0.
\end{equation*}

Negative XS cause instabilities in transport sweeps.

\subsection*{Reformulation as Standard Eigenvalue Problem}

Rewrite discretized form of $\alpha$-NTE:

\begin{equation*}
(\alpha \mathbf{L}_{\alpha} + \mathbf{L})\psi^{\alpha} = \mathbf{M}(\mathbf{S}+\mathbf{F})\mathbf{D}\psi^{\alpha}
\end{equation*}

as

\begin{equation*}
\alpha \psi^{\alpha} = \mathbf{L}_{\alpha}^{-1} [\mathbf{M}(\mathbf{S}+\mathbf{F})\mathbf{D} - \mathbf{L}]\psi^{\alpha}
\end{equation*}

and use standard eigensolvers for subcritical systems ($\alpha < 0$) for which dominant $\alpha$ is smallest in magnitude. For supercritical systems, need correction $\frac{\beta}{v}\psi^{\alpha}$ that is subtracted from both sides:

\begin{equation*}
\alpha \psi^{\alpha} - \frac{\beta}{v}\psi^{\alpha} = \mathbf{L}_{\alpha}^{-1} [\mathbf{M}(\mathbf{S}+\mathbf{F})\mathbf{D} - \mathbf{L}]\psi^{\alpha} - \frac{\beta}{v}\psi^{\alpha}
\end{equation*}

Effective eigenvalues, $(\alpha - \beta)$, are all negative though method requires prior knowledge of eigenvalue to determine correction term. Also problematic since codes usually apply $\mathbf{L}^{-1}$ instead of $\mathbf{L}$ directly.

\subsection*{Recast into Nonlinear Iteration Problem}

Eigenvalue problems can be thought of nonlinear problems since they involve product of two unknowns, the eigenvalue and eigenvector.

For $\alpha$-NTE, unknown is vector $\mathbf{x} = \begin{bmatrix} \mathbf{y} \\ \alpha \end{bmatrix}$ 

Fixed-point iteration:

\begin{equation*}
\mathbf{y}^{\ell + 1} = \mathbf{P}(\alpha^{(\ell)})\mathbf{y}^{\ell}
\end{equation*}
\begin{equation*}
\alpha^{\ell + 1} = \alpha^{(\ell)} + a(\mathbf{x}^{\ell})
\end{equation*}

Can be rewritten as $\mathbf{x}^{(\ell + 1)} = \mathbf{x}^{(\ell)} - \mathcal{F}(\mathbf{x}^{(\ell)})$ where $\mathcal{F}$ is a nonlinear residual given by 

\begin{equation*}
\mathcal{F}(\mathbf{x}) = \begin{bmatrix} \mathbf{y} - \mathbf{P}(\alpha)\mathbf{y} \\ -a(\mathbf{x}) \end{bmatrix}
\end{equation*}

$\mathbf{y}$ represents either $\phi^{\alpha}$ or $\psi^{\alpha}$ depending on the formulation and $a$ is the eigenvalue update.

\end{document}