\documentclass{article}
\usepackage[utf8]{inputenc}
\usepackage{notes}

% In your .tex file
% !TEX program = pdflatex

\newcommand{\thiscoursecode}{[XXXX] (\#\#\#)}
\newcommand{\thiscoursename}{Alpha-Eigenvalue Research at UC Berkeley}
\newcommand{\thisprof}{Dr. Great Professor}
\newcommand{\me}{Liam Horne}
\newcommand{\thisterm}{Paper Summary}
\newcommand{\website}{MYWEBSITE.COM}

% Headers
\chead{\thiscoursename}
\lhead{\thisterm}


%%%%%% TITLE %%%%%%
\newcommand{\notefront} {
\pagenumbering{roman}
\begin{center}

{\ttfamily \url{\website}} {\small}

\textbf{\Huge{\noun{\thiscoursecode}}}{\Huge \par}

{\large{\noun{\thiscoursename}}}\\ \vspace{0.1in}

  {\noun \thisprof} \ $\bullet$ \ {\noun \thisterm} \ $\bullet$ \ {\noun {University of Waterloo}} \\

  \end{center}
  }
  
  \title{Time-absorption Eigenvalue Searches using Diffusion Synthetic Acceleration}
  
\date{\vspace{-5ex}}
%\date{}

\begin{document}

\maketitle

\section*{Information}

Authors: J. A. Dahl and R. E. Alcouffe \\

Organization: CCS-4, Los Alamos National Laboratory

\section*{Introduction}

Diffusion Synthetic Acceleration (DSA) used to accelerate inner (scattering) and outer (fission) source iterations. Time-absorption eigenvalue transport equation:

\begin{equation*}
L \psi + \sigma_t \psi + \frac{\alpha}{v} \psi = \frac{1}{k_{\text{eff}}} \nu \sigma_f \phi.
\end{equation*}

For critical system ($\alpha = 0$), $k_{\text{eff}}$ solved by PI. Define

\begin{equation*}
\lambda^k = \frac{ \langle \nu \sigma_f \phi \rangle ^{k}}{ \langle \nu \sigma_f \phi \rangle ^{k-1} }
\end{equation*}

and update $k_{\text{eff}}^{k} = \lambda_k k_{\text{eff}}^{k-1}$. Procedure continues until desired convergence criterion is met. In outer DSA method, the multigroup transport operator is replaced by MG diffusion corrected DSA equation, increasing time efficiency of solution.

\section*{DSA Alpha Search Algorithm}

$\alpha$-eigenvalue problem solved within $k_{\text{eff}}$ PI. $k_{\text{eff}}$ set to constant (usually $k_{\text{eff}}=1$) with $\alpha$ estimated by some value. Iterations performed as before with a new $k_{\text{eff}}$ calculated. $\alpha$ is updated to achieve a value of $\lambda^{k+1}$ closer to 1. New method: take two iterative states where $\alpha$ is nonzero, then

\begin{equation*}
(\alpha^{k+1} - \alpha^k) \bigg \langle \frac{\phi}{v} \bigg \rangle^{k+1} = \langle \nu \sigma_f \phi \rangle^{k+1} - \langle \nu \sigma_f \phi \rangle^{k}.
\end{equation*}

We update $\alpha$ using

\begin{equation*}
\alpha^{k+1} = \alpha^{k} + d \bigg [ (\lambda^k - 1) \frac{\langle \nu \sigma_f \phi \rangle^{k}}{\bigg \langle \frac{\phi}{v} \bigg \rangle^{k}} \bigg ]
\end{equation*}

where $d$ damping parameter used in early stages to prevent additive term from being much larger compared with $\alpha^k$. Without damping parameter, effective removal term of the transport operator can become negative. 

\end{document}