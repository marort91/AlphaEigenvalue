\documentclass{article}
\usepackage[utf8]{inputenc}
\usepackage{notes}

% In your .tex file
% !TEX program = pdflatex

\newcommand{\thiscoursecode}{[XXXX] (\#\#\#)}
\newcommand{\thiscoursename}{Alpha-Eigenvalue Research at UC Berkeley}
\newcommand{\thisprof}{Dr. Great Professor}
\newcommand{\me}{Liam Horne}
\newcommand{\thisterm}{Paper Summary}
\newcommand{\website}{MYWEBSITE.COM}

% Headers
\chead{\thiscoursename}
\lhead{\thisterm}


%%%%%% TITLE %%%%%%
\newcommand{\notefront} {
\pagenumbering{roman}
\begin{center}

{\ttfamily \url{\website}} {\small}

\textbf{\Huge{\noun{\thiscoursecode}}}{\Huge \par}

{\large{\noun{\thiscoursename}}}\\ \vspace{0.1in}

  {\noun \thisprof} \ $\bullet$ \ {\noun \thisterm} \ $\bullet$ \ {\noun {University of Waterloo}} \\

  \end{center}
  }
  
  \title{Computation of Fundamental Time-eigenvalue of the Neutron Transport Equation}
  
\date{\vspace{-5ex}}
%\date{}

\begin{document}

\maketitle

\section*{Information}

Authors: Huayun Shen, Bin Zhong, and Huipo Liu \\

Organization: Institute of Applied Physics and Computational Mathematics, Department 6

\section*{Methods}

\subsection*{Modified $\alpha$-k Power Iteration Method}

\begin{equation*}
\begin{split}
\hat{\Omega} \cdot \nabla (\phi_{\alpha} - \phi_{k}) + \Sigma_{t}(\phi_{\alpha} - \phi_{k}) + \frac{\alpha}{v}\phi_{\alpha} = \int_{0}^{\infty}dE' \int_{4 \pi} d\Omega' \Sigma_{s} (\vec{r},E' \rightarrow E, \hat{\Omega}' \cdot \hat{\Omega}) (\phi_{\alpha}(\vec{r},E',\hat{\Omega}')-\phi_{k}(\vec{r},E',\hat{\Omega}')) + \\ \frac{\chi(E)}{4 \pi} \int_{0}^{\infty} dE' \nu_p \Sigma_f \int_{4 \pi} d\hat{\Omega}'(\phi_{\alpha}(\vec{r},E',\hat{\Omega}') - \frac{\phi_{k}(\vec{r},E',\hat{\Omega}')}{k}
\end{split}
\end{equation*}

Assume $\phi_{\alpha} \approx \phi_{k}$. The previous expression simplifies to

\begin{equation*}
\alpha \approx \frac{1}{\phi_k/v}\frac{k-1}{k}\frac{\chi(E)}{4 \pi} \int_{0}^{\infty} dE' \nu_p \Sigma_f \int_{4 \pi} d\Omega' \phi_{k}(\vec{r},E',\hat{\Omega}')
\end{equation*}


$\alpha$ can be estimated at any point where $\Sigma_f \neq 0$. \\

Using the following relationship:

\begin{equation*}
\int_{0}^{\infty} dE \int_{4\pi} d\Omega \frac{\chi(E)}{4\pi} = 1
\end{equation*}

we can approximate $\alpha$ as 

\begin{equation*}
\alpha \approx \frac{k-1}{k} \frac{\int_{0}^{\infty} dE' \nu_p \Sigma_f \int_{4\pi} d\hat{\Omega}' \phi_{k}(\vec{r},E',\hat{\Omega}')}{\int_{0}^{\infty}\int_{4\pi} d\hat{\Omega}\frac{\phi_k}{v}}
\end{equation*}
\\
If problem has symmetry, better to calculate $\alpha$ at central point if there is fissionable material at that point.

Better estimate by integrating over everywhere where there is fissionable material:

\begin{equation*}
\alpha \approx \frac{k-1}{k} \frac{\int_{\Sigma_f \neq 0} dV \int_{0}^{\infty} dE' \nu_p \Sigma_f \int_{4\pi} d\hat{\Omega}' \phi_{k}(\vec{r},E',\hat{\Omega}')}{\int_{\Sigma_f \neq 0} dV \int_{0}^{\infty}\int_{4\pi} d\hat{\Omega}\frac{\phi_k}{v}}
\end{equation*}

\subsubsection*{Subcritical Problem $\alpha$ < 0}

Method is unstable when $\alpha$ < 0 (i.e. when the system is subcritical). $\alpha$-eigenvalue equation rewritten with an arbitrary parameter $\eta$ that is greater than zero.

\begin{equation*}
\begin{split}
\hat{\Omega} \cdot \phi_{\alpha} + \bigg ( \Sigma_t - \eta \frac{\alpha}{v} \bigg ) \phi_{\alpha} = \int_{0}^{\infty} dE' \int_{4\pi} d\hat{\Omega}' \Sigma_{s}(\vec{r},E'\rightarrow E, \hat{\Omega}' \cdot \hat{\Omega})\phi_{\alpha}(\vec{r},E',\hat{\Omega}') + \\ \frac{1}{k}\frac{\chi(E)}{4\pi} \int_{0}^{\infty} dE' \nu_{p} \Sigma_{f}(\vec{r},E') \int_{4\pi} d\hat{\Omega}'\phi_{\alpha}(\vec{r},E',\hat{\Omega}' - (1+\eta)\frac{\alpha}{v}\phi_{\alpha}(\vec{r},E',\hat{\Omega}')
\end{split}
\end{equation*}

Default is usually $\eta$ = 1.

\subsection*{Modified Iterative Method}

Iterative Method Algorithm:

\begin{enumerate}

\item Solve k-effective TE with $\alpha$ = 0 for $k_0$ and $\phi_0$.
\item Estimate the value $\alpha_{n-1/2}$ from $\alpha$ expressions seen above.
\item Let $\alpha_n = \alpha_{n-1} + \alpha_{n-1/2}$ and solve $\alpha$-NTE to find new $k_n$ and $\phi_n$.
\item Repeat steps 2 and 3 until $k_n$ = 1. $\alpha_n$ will be desired eigenvalue.

\end{enumerate}


\end{document}
