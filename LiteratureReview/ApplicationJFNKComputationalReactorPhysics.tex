\documentclass{article}
\usepackage[utf8]{inputenc}
\usepackage{notes}

% In your .tex file
% !TEX program = pdflatex

\newcommand{\thiscoursecode}{[XXXX] (\#\#\#)}
\newcommand{\thiscoursename}{Alpha-Eigenvalue Research at UC Berkeley}
\newcommand{\thisprof}{Dr. Great Professor}
\newcommand{\me}{Liam Horne}
\newcommand{\thisterm}{Paper Summary}
\newcommand{\website}{MYWEBSITE.COM}

% Headers
\chead{\thiscoursename}
\lhead{\thisterm}


%%%%%% TITLE %%%%%%
\newcommand{\notefront} {
\pagenumbering{roman}
\begin{center}

{\ttfamily \url{\website}} {\small}

\textbf{\Huge{\noun{\thiscoursecode}}}{\Huge \par}

{\large{\noun{\thiscoursename}}}\\ \vspace{0.1in}

  {\noun \thisprof} \ $\bullet$ \ {\noun \thisterm} \ $\bullet$ \ {\noun {University of Waterloo}} \\

  \end{center}
  }
  
  \title{Application of the Jacobian-Free Newton-Krylov Method in Computational Reactor Physics}
  
\date{\vspace{-5ex}}
%\date{}

\begin{document}

\maketitle

\section*{Information}

Authors: D. A. Knoll, H. Park, and Kord Smith \\

Organization: Idaho National Laboratory, Multiphysics Methods Group and Studvisk Scandpower

\section*{Introduction}

JFNK method is combination of Newton's method for nonlinear iteration and Krylov-based linear iterative methods. Two unique features: the ability to perform a Newton iteration on a complicated multiscale function where forming the Jacobian is not a good idea and the ability to use another solver for the same problem as a preconditioning method. Paper considers two applications: JFNK to execute Newton iteration for nonlinear diffusion acceleration and using standard k-eigenvalue iteration inside the JFNK iteration. k becomes part of the Krylov solution vector along with flux and a normalized eigenvalue update can be used as the additional equations. Power method as preconditioner also considered.

\section*{Algorithm}

Newton's method solves problem

\begin{equation*}
\mathbf{F}(\phi) = 0
\end{equation*}

using iteration

\begin{equation*}
\mathbf{J}^k \delta \phi^k = -\mathbf{\phi}(\phi^k)
\end{equation*}

\begin{equation*}
\phi^{k+1} = \phi^k + d * \delta \phi^k
\end{equation*}

If we use Krylov method (GMRES), we don't form Jacobian explicitly and approximate the action of the Jacobian on a vector by a first-order Taylor series expansion

\begin{equation*}
\mathbf{Jv} \approx \frac{\mathbf{F}(\phi + \epsilon \mathbf{v}) - \mathbf{F}(\phi)}{\epsilon}
\end{equation*}

\section*{Precondtioning}

Key to efficient application of JFNK in typical computational problems is preconditioning. Goal of preconditioning to cluster eigenvalues  which limits Krylov iterations per Newton iteration. Right preconditioning of linear problem expressed with $\mathbf{M}$ is the preconditioning operator and the inverse the preconditioning process 

\begin{equation*}
\mathbf{J}^k \mathbf{M}^{-1} \mathbf{M}  \delta \phi^k = -\mathbf{\phi}(\phi^k)
\end{equation*}

Applying to matrix-vector multiply

\begin{equation*}
\mathbf{JM^{-1}v} \approx \frac{\mathbf{F}(\phi + \epsilon \mathbf{M}^{-1} \mathbf{v}) - \mathbf{F}(\phi)}{\epsilon}
\end{equation*}.

Actually done in two steps: Perform $\mathbf{y} = \mathbf{M}^{-1} \mathbf{v}$ and then perform on $\mathbf{Jy} \approx [\mathbf{F}(\phi + \epsilon \mathbf{y}) - \mathbf{F}(\phi)] / \epsilon$.

\section*{Nonlinear Acceleration}

1-D mono-energetic transport equation in homogeneous slab geometry ($0 \leq x \leq L$) with isotropic scattering ($-1 \leq \mu \leq 1$) and fixed source:

\begin{equation*}
\mu \frac{\partial}{\partial x} \psi(x,\mu) + \Sigma_t \psi(x,\mu) = \frac{1}{2}(\Sigma_s \phi(x) + Q).
\end{equation*}

Nonlinear acceleration methods utilize zeroth moment of transport equation in angle (balance statement):

\begin{equation*}
\frac{dJ}{dx} + ( \Sigma_t - \Sigma_s) \phi = Q.
\end{equation*}

Methods vary based on their closure for J and whether or no they solve for a correction or a new solution to the scalar flux.

To be continued on acceleration methods. Moved onto k-eigenvalue iteration.

\section*{k-eigenvalue Iteration}

Consider one group, NDE criticality problem:

\begin{equation*}
\frac{d}{dx} \bigg [ - \frac{1}{3 \Sigma_t} \frac{ d \phi}{dx} \bigg ] + \Sigma_a \phi = \frac{\nu}{k} \Sigma_f \phi.
\end{equation*}

Power iteration solves fixed source method

\begin{equation*}
\frac{d}{dx} \bigg [ - \frac{1}{3 \Sigma_t} \frac{ d \phi^{n+1}}{dx} \bigg ] + \Sigma_a \phi^{n+1} = \frac{\nu}{k} \Sigma_f \phi^{n}.
\end{equation*}

that uses updated solution to approximate a new k value

\begin{equation*}
k^{n+1} = \frac{k^n \sum_{grid} (\nu \Sigma_f \phi^{n+1})}{\sum_{grid}(\nu \Sigma_f \phi^n)}
\end{equation*}

Recast problem into nonlinear system $[F_\phi,F_k]^T = 0$ to be solved using fully implicit JFNK

\begin{equation*}
F_\phi = \frac{d}{dx} \bigg [ - \frac{1}{3 \Sigma_t} \frac{ d \phi^{n+1}}{dx} \bigg ] + \Sigma_a \phi^{n+1} - \frac{\nu}{k} \Sigma_f \phi^{n+1}.
\end{equation*}

and

\begin{equation*}
F_k = k^{n+1} - \sum(\nu \Sigma_f \phi^{n+1})
\end{equation*}

with normalization 

\begin{equation*}
\frac{k^n}{\sum (\nu \Sigma_f \phi^n)} = 1
\end{equation*}

Unknowns in Krylov vector are fluxes at each cell, $\phi_i$, and eigenvalue k. Newton iteration system is

\[
\begin{bmatrix}
J_{\phi,\phi} & J_{\phi,k} \\
J_{k,\phi} & J_{k,k}
\end{bmatrix}
\begin{bmatrix}
\delta \phi \\ \delta k
\end{bmatrix}
= 
-\begin{bmatrix}
F_{\phi} \\ F_{k}
\end{bmatrix}
\]

PI used as preconditioner. Preconditioner is lower block triangular

\[
M = 
\begin{bmatrix}
M_{\phi,\phi} & 0 \\ 
M_{k,\phi} & M_{k,k}
\end{bmatrix}
\]

with $M_{\phi,\phi} = \frac{d}{dx} \bigg [ -\frac{1}{3\Sigma_t} \frac{d}{dx} \bigg ] + \Sigma_a$, $M_{k,\phi} = -\sum \nu \Sigma_f, M_{k,k} = 1$.

Preconditioning process equivalent to iteration of PI in residual form. Preconditioner inverse has two steps

\begin{equation*}
\delta \phi = M^{-1}_{\phi,\phi}(F_{\phi}) \text{ (flux solve})
\end{equation*}

\begin{equation*}
\delta k = M^{-1}_{k,k}(-F_k - M_{k,\phi} \delta \phi) \text{ (eigenvalue update)}.
\end{equation*}

\end{document}