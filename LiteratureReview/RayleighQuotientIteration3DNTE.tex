\documentclass{article}
\usepackage[utf8]{inputenc}
\usepackage{notes}

% In your .tex file
% !TEX program = pdflatex

\newcommand{\thiscoursecode}{[XXXX] (\#\#\#)}
\newcommand{\thiscoursename}{Alpha-Eigenvalue Research at UC Berkeley}
\newcommand{\thisprof}{Dr. Great Professor}
\newcommand{\me}{Liam Horne}
\newcommand{\thisterm}{Paper Summary}
\newcommand{\website}{MYWEBSITE.COM}

% Headers
\chead{\thiscoursename}
\lhead{\thisterm}


%%%%%% TITLE %%%%%%
\newcommand{\notefront} {
\pagenumbering{roman}
\begin{center}

{\ttfamily \url{\website}} {\small}

\textbf{\Huge{\noun{\thiscoursecode}}}{\Huge \par}

{\large{\noun{\thiscoursename}}}\\ \vspace{0.1in}

  {\noun \thisprof} \ $\bullet$ \ {\noun \thisterm} \ $\bullet$ \ {\noun {University of Waterloo}} \\

  \end{center}
  }
  
  \title{Rayleigh Quotient Iteration in 3D, Deterministic Neutron Transport}
  
\date{\vspace{-5ex}}
%\date{}

\begin{document}

\maketitle

\section*{Information}

Authors: R. N. Slaybaugh, T. M. Evans, G. G. Davidson, and P. P. H. Wilson \\

Organization: Department of Nuclear Engineering and Engineering Physics, University of Wisconsin, Madison and Radiation Transport Group, Oak Ridge National Laboratory

\section*{Background}

The multigroup (MG) discrete ordinates $S_N$ equations written in operator form:

\begin{equation*}
\mathbf{L} \psi = \mathbf{MS} \phi + \frac{1}{k}\mathbf{M} \chi f^{T} \phi
\end{equation*}

where $\psi$ is the angular flux, $\phi$ the angular flux moments, $\mathbf{L}$ is the first-order linear differential transport operator, $\mathbf{M}$ is the moment-to-discrete operator, $\mathbf{S}$ is the scattering operator and everything else is related to fission. Problem can be converted to fixed source problem by replacing fission source term on right hand side of the equation by $q_e$.

Angular flux can be related to angular flux moments through the discrete-to-moment operator $\mathbf{D}$ by

\begin{equation*}
\phi = \mathbf{D} \psi
\end{equation*}

We combine the previous two equations to have one equation for $\phi$ and define $\mathbf{T} = \mathbf{DL}^{-1}$ and $\mathbf{F} = \chi f^{T}$ to obtain

\begin{equation*}
(\mathbf{I} - \mathbf_{TMS}) \phi = \frac{1}{k}\mathbf{TMF}\phi
\end{equation*}

\subsection*{Linear Algebra Preliminaries}

\begin{itemize}

\item The spectrum of $\mathbf{A}$ is defined as $\sigma(\mathbf{A}) \equiv \{ \lambda \in \mathbb{C}: rank(\mathbf{A} - \lambda \mathbf{I}) \leq n \}$ with the eigenvalues ordered as $\lvert\lambda_1 \rvert \geq \lvert \lambda_2 \rvert \geq \ldots \geq \lvert \lambda_n \rvert$.

\item The dominance ratio (DR) is defined as $ DR \equiv \lambda_1 / \lambda_2$ for an $n \times n$ matrix $\mathbf{A}$ whose eigenvectors and eigenvalues satisfy

\begin{equation*}
\mathbf{A} x_i = \lambda_i x_i
\end{equation*}

for $i = 1,\ldots,n$.

\subsection*{Power Iteration}

Usually use Power Iteration (PI) to solve for eigenvalue. Attractive method since it only requires matrix-vector products and storage of two vectors. PI uses ordinary form ($\phi$,k) as opposed to generalized form of the NTE eigenvalue-eigenvector pair ($\phi$,$\frac{1}{k}$). In legacy codes, eigenvector is usually fission vector. PI algorithm:

\begin{equation*}
\begin{split}
\mathbf{A}\phi = k \phi \\
\text{where } \mathbf{A} = (\mathbf{I} - \mathbf{TMS})^{-1}\mathbf{TMF}, \\
\phi^{i+1} = \frac{1}{k^i} \mathbf{A} \phi^{i};\text{			}k^{i+1} = k^{i} \frac{f^T \phi^{i+1}}{f^{T}\phi^i}
\end{split}
\end{equation*}

Inside PI requires inner iterations to solve the MG problem (looks like a fixed source problem):

\begin{equation*}
(\mathbf{I} - \mathbf{TMS}) y^i = \mathbf{TMF}\phi^i
\end{equation*}

PI convergence can be slow due to DR for highly coupled problems. 

\subsection*{Shifted Inverse Iteration}

Shifted Inverse Iteration (SII) typically converges more quickly than PI. For some shift $\mu$, it can be shown that $(\mathbf{A} - \mu \mathbf{I})$ has the same eigenvalues as $\mathbf{A}$. If $\mu \not\in \sigma(\mathbf{A})$, then $(\mathbf{A} - \mu \mathbf{I})$ is invertible. If invertible, then

\begin{equation*}
\sigma([\mathbf{A} - \mu \mathbf{I}]^{-1}) = \{ 1 / (\lambda - \mu) : \lambda \in \sigma(\mathbf{A})\}.
\end{equation*}

If the shift is good ($ \mu \approx \lambda_1$), then the eigenvalues near shift will become extremal values. We then use PI on the shifted matrix. 

\end{document}